% Options for packages loaded elsewhere
\PassOptionsToPackage{unicode}{hyperref}
\PassOptionsToPackage{hyphens}{url}
\PassOptionsToPackage{dvipsnames,svgnames,x11names}{xcolor}
%
\documentclass[
  letterpaper,
  DIV=11,
  numbers=noendperiod]{scrartcl}

\usepackage{amsmath,amssymb}
\usepackage{iftex}
\ifPDFTeX
  \usepackage[T1]{fontenc}
  \usepackage[utf8]{inputenc}
  \usepackage{textcomp} % provide euro and other symbols
\else % if luatex or xetex
  \usepackage{unicode-math}
  \defaultfontfeatures{Scale=MatchLowercase}
  \defaultfontfeatures[\rmfamily]{Ligatures=TeX,Scale=1}
\fi
\usepackage{lmodern}
\ifPDFTeX\else  
    % xetex/luatex font selection
\fi
% Use upquote if available, for straight quotes in verbatim environments
\IfFileExists{upquote.sty}{\usepackage{upquote}}{}
\IfFileExists{microtype.sty}{% use microtype if available
  \usepackage[]{microtype}
  \UseMicrotypeSet[protrusion]{basicmath} % disable protrusion for tt fonts
}{}
\makeatletter
\@ifundefined{KOMAClassName}{% if non-KOMA class
  \IfFileExists{parskip.sty}{%
    \usepackage{parskip}
  }{% else
    \setlength{\parindent}{0pt}
    \setlength{\parskip}{6pt plus 2pt minus 1pt}}
}{% if KOMA class
  \KOMAoptions{parskip=half}}
\makeatother
\usepackage{xcolor}
\setlength{\emergencystretch}{3em} % prevent overfull lines
\setcounter{secnumdepth}{5}
% Make \paragraph and \subparagraph free-standing
\ifx\paragraph\undefined\else
  \let\oldparagraph\paragraph
  \renewcommand{\paragraph}[1]{\oldparagraph{#1}\mbox{}}
\fi
\ifx\subparagraph\undefined\else
  \let\oldsubparagraph\subparagraph
  \renewcommand{\subparagraph}[1]{\oldsubparagraph{#1}\mbox{}}
\fi

\usepackage{color}
\usepackage{fancyvrb}
\newcommand{\VerbBar}{|}
\newcommand{\VERB}{\Verb[commandchars=\\\{\}]}
\DefineVerbatimEnvironment{Highlighting}{Verbatim}{commandchars=\\\{\}}
% Add ',fontsize=\small' for more characters per line
\usepackage{framed}
\definecolor{shadecolor}{RGB}{241,243,245}
\newenvironment{Shaded}{\begin{snugshade}}{\end{snugshade}}
\newcommand{\AlertTok}[1]{\textcolor[rgb]{0.68,0.00,0.00}{#1}}
\newcommand{\AnnotationTok}[1]{\textcolor[rgb]{0.37,0.37,0.37}{#1}}
\newcommand{\AttributeTok}[1]{\textcolor[rgb]{0.40,0.45,0.13}{#1}}
\newcommand{\BaseNTok}[1]{\textcolor[rgb]{0.68,0.00,0.00}{#1}}
\newcommand{\BuiltInTok}[1]{\textcolor[rgb]{0.00,0.23,0.31}{#1}}
\newcommand{\CharTok}[1]{\textcolor[rgb]{0.13,0.47,0.30}{#1}}
\newcommand{\CommentTok}[1]{\textcolor[rgb]{0.37,0.37,0.37}{#1}}
\newcommand{\CommentVarTok}[1]{\textcolor[rgb]{0.37,0.37,0.37}{\textit{#1}}}
\newcommand{\ConstantTok}[1]{\textcolor[rgb]{0.56,0.35,0.01}{#1}}
\newcommand{\ControlFlowTok}[1]{\textcolor[rgb]{0.00,0.23,0.31}{#1}}
\newcommand{\DataTypeTok}[1]{\textcolor[rgb]{0.68,0.00,0.00}{#1}}
\newcommand{\DecValTok}[1]{\textcolor[rgb]{0.68,0.00,0.00}{#1}}
\newcommand{\DocumentationTok}[1]{\textcolor[rgb]{0.37,0.37,0.37}{\textit{#1}}}
\newcommand{\ErrorTok}[1]{\textcolor[rgb]{0.68,0.00,0.00}{#1}}
\newcommand{\ExtensionTok}[1]{\textcolor[rgb]{0.00,0.23,0.31}{#1}}
\newcommand{\FloatTok}[1]{\textcolor[rgb]{0.68,0.00,0.00}{#1}}
\newcommand{\FunctionTok}[1]{\textcolor[rgb]{0.28,0.35,0.67}{#1}}
\newcommand{\ImportTok}[1]{\textcolor[rgb]{0.00,0.46,0.62}{#1}}
\newcommand{\InformationTok}[1]{\textcolor[rgb]{0.37,0.37,0.37}{#1}}
\newcommand{\KeywordTok}[1]{\textcolor[rgb]{0.00,0.23,0.31}{#1}}
\newcommand{\NormalTok}[1]{\textcolor[rgb]{0.00,0.23,0.31}{#1}}
\newcommand{\OperatorTok}[1]{\textcolor[rgb]{0.37,0.37,0.37}{#1}}
\newcommand{\OtherTok}[1]{\textcolor[rgb]{0.00,0.23,0.31}{#1}}
\newcommand{\PreprocessorTok}[1]{\textcolor[rgb]{0.68,0.00,0.00}{#1}}
\newcommand{\RegionMarkerTok}[1]{\textcolor[rgb]{0.00,0.23,0.31}{#1}}
\newcommand{\SpecialCharTok}[1]{\textcolor[rgb]{0.37,0.37,0.37}{#1}}
\newcommand{\SpecialStringTok}[1]{\textcolor[rgb]{0.13,0.47,0.30}{#1}}
\newcommand{\StringTok}[1]{\textcolor[rgb]{0.13,0.47,0.30}{#1}}
\newcommand{\VariableTok}[1]{\textcolor[rgb]{0.07,0.07,0.07}{#1}}
\newcommand{\VerbatimStringTok}[1]{\textcolor[rgb]{0.13,0.47,0.30}{#1}}
\newcommand{\WarningTok}[1]{\textcolor[rgb]{0.37,0.37,0.37}{\textit{#1}}}

\providecommand{\tightlist}{%
  \setlength{\itemsep}{0pt}\setlength{\parskip}{0pt}}\usepackage{longtable,booktabs,array}
\usepackage{calc} % for calculating minipage widths
% Correct order of tables after \paragraph or \subparagraph
\usepackage{etoolbox}
\makeatletter
\patchcmd\longtable{\par}{\if@noskipsec\mbox{}\fi\par}{}{}
\makeatother
% Allow footnotes in longtable head/foot
\IfFileExists{footnotehyper.sty}{\usepackage{footnotehyper}}{\usepackage{footnote}}
\makesavenoteenv{longtable}
\usepackage{graphicx}
\makeatletter
\def\maxwidth{\ifdim\Gin@nat@width>\linewidth\linewidth\else\Gin@nat@width\fi}
\def\maxheight{\ifdim\Gin@nat@height>\textheight\textheight\else\Gin@nat@height\fi}
\makeatother
% Scale images if necessary, so that they will not overflow the page
% margins by default, and it is still possible to overwrite the defaults
% using explicit options in \includegraphics[width, height, ...]{}
\setkeys{Gin}{width=\maxwidth,height=\maxheight,keepaspectratio}
% Set default figure placement to htbp
\makeatletter
\def\fps@figure{htbp}
\makeatother
% definitions for citeproc citations
\NewDocumentCommand\citeproctext{}{}
\NewDocumentCommand\citeproc{mm}{%
  \begingroup\def\citeproctext{#2}\cite{#1}\endgroup}
\makeatletter
 % allow citations to break across lines
 \let\@cite@ofmt\@firstofone
 % avoid brackets around text for \cite:
 \def\@biblabel#1{}
 \def\@cite#1#2{{#1\if@tempswa , #2\fi}}
\makeatother
\newlength{\cslhangindent}
\setlength{\cslhangindent}{1.5em}
\newlength{\csllabelwidth}
\setlength{\csllabelwidth}{3em}
\newenvironment{CSLReferences}[2] % #1 hanging-indent, #2 entry-spacing
 {\begin{list}{}{%
  \setlength{\itemindent}{0pt}
  \setlength{\leftmargin}{0pt}
  \setlength{\parsep}{0pt}
  % turn on hanging indent if param 1 is 1
  \ifodd #1
   \setlength{\leftmargin}{\cslhangindent}
   \setlength{\itemindent}{-1\cslhangindent}
  \fi
  % set entry spacing
  \setlength{\itemsep}{#2\baselineskip}}}
 {\end{list}}
\usepackage{calc}
\newcommand{\CSLBlock}[1]{\hfill\break\parbox[t]{\linewidth}{\strut\ignorespaces#1\strut}}
\newcommand{\CSLLeftMargin}[1]{\parbox[t]{\csllabelwidth}{\strut#1\strut}}
\newcommand{\CSLRightInline}[1]{\parbox[t]{\linewidth - \csllabelwidth}{\strut#1\strut}}
\newcommand{\CSLIndent}[1]{\hspace{\cslhangindent}#1}

\KOMAoption{captions}{tableheading}
\makeatletter
\@ifpackageloaded{caption}{}{\usepackage{caption}}
\AtBeginDocument{%
\ifdefined\contentsname
  \renewcommand*\contentsname{Table of contents}
\else
  \newcommand\contentsname{Table of contents}
\fi
\ifdefined\listfigurename
  \renewcommand*\listfigurename{List of Figures}
\else
  \newcommand\listfigurename{List of Figures}
\fi
\ifdefined\listtablename
  \renewcommand*\listtablename{List of Tables}
\else
  \newcommand\listtablename{List of Tables}
\fi
\ifdefined\figurename
  \renewcommand*\figurename{Figure}
\else
  \newcommand\figurename{Figure}
\fi
\ifdefined\tablename
  \renewcommand*\tablename{Table}
\else
  \newcommand\tablename{Table}
\fi
}
\@ifpackageloaded{float}{}{\usepackage{float}}
\floatstyle{ruled}
\@ifundefined{c@chapter}{\newfloat{codelisting}{h}{lop}}{\newfloat{codelisting}{h}{lop}[chapter]}
\floatname{codelisting}{Listing}
\newcommand*\listoflistings{\listof{codelisting}{List of Listings}}
\makeatother
\makeatletter
\makeatother
\makeatletter
\@ifpackageloaded{caption}{}{\usepackage{caption}}
\@ifpackageloaded{subcaption}{}{\usepackage{subcaption}}
\makeatother
\ifLuaTeX
  \usepackage{selnolig}  % disable illegal ligatures
\fi
\usepackage{bookmark}

\IfFileExists{xurl.sty}{\usepackage{xurl}}{} % add URL line breaks if available
\urlstyle{same} % disable monospaced font for URLs
\hypersetup{
  pdftitle={The Anatomy of a Hit: Statistically Learning from the Best},
  pdfauthor={Luca Carnegie},
  colorlinks=true,
  linkcolor={blue},
  filecolor={Maroon},
  citecolor={Blue},
  urlcolor={Blue},
  pdfcreator={LaTeX via pandoc}}

\title{The Anatomy of a Hit: Statistically Learning from the
Best\thanks{Code and data are available at:
https://github.com/lcarnegie/popularity-modeling.}}
\usepackage{etoolbox}
\makeatletter
\providecommand{\subtitle}[1]{% add subtitle to \maketitle
  \apptocmd{\@title}{\par {\large #1 \par}}{}{}
}
\makeatother
\subtitle{Modeling the most impactful elements of the most popular
musicians.}
\author{Luca Carnegie}
\date{April 18, 2024}

\begin{document}
\maketitle
\begin{abstract}
We attempt to do some interesting things with models and music - stay
tuned!
\end{abstract}

\renewcommand*\contentsname{Table of contents}
{
\hypersetup{linkcolor=}
\setcounter{tocdepth}{3}
\tableofcontents
}
\newpage

\section{Introduction}\label{introduction}

Throughout the history of the music industry, musicians successful or
not have always pined for one thing - writing their first hit and making
it big. However, for large music firms, taking on musicians that have
yet to generate considerable popularity is a large risk, since they are
often gambling on an unknown je ne sais quoi, and often, any given
artist is more likely to not produce a hit than produce a hit; this is
easily shown through the presence of a few small ``superstars'' that
consistently dominate the music market (Rosen 1981). This poses an
interesting question for artists vying to become that superstar: what
does make a song a ``hit''?

One way to measure how popular a song is through the `valence' or
perceived energy of a song.{[}investigate the motivation here{]}.
Measuring valence has been done at a national and cross-artistic-work
level (Dodds and Danforth 2010), but I investigate solely song valence
on an individual basis. Though difficult to exactly predict what
individual song could be a hit, it is very possible to wean out insight
from previously released music to statistically learn the most important
elements of a great song.

Using data from Spotify's API, I construct a linear model of valence or
``energy'', using the discographies of the ten most popular artists
{[}find a source{]} of every particular genre in every decade since
1950. I find that {[}there are interesting things to be found\ldots{]}.
Blah Blah Blah.

In the streaming era, competition for stardom is at an all-time high.
Being cognizant of the most impactful elements could encourage artists
to focus on what truly works to generate hits. This may allow them to
carve their own paths outside of the controlling nature of a record
contract (Burke 1997). These insights could be used to give more
knowledge and freedom to artists to be able to strike out on their own
and carve their own destinies in the world of music.

\section{Data}\label{sec-data}

\section{Model}\label{model}

\subsection{Model set-up}\label{model-set-up}

\subsubsection{Model justification}\label{model-justification}

\section{Results}\label{results}

\section{Discussion}\label{discussion}

\subsection{First discussion point}\label{sec-first-point}

\subsection{Second discussion point}\label{sec-second-point}

\subsection{Third discussion point}\label{sec-third-point}

\subsection{Weaknesses and next steps}\label{sec-weaknesses-next}

Weaknesses and next steps should also be included.

\newpage

\appendix

\section*{Appendix}\label{appendix}
\addcontentsline{toc}{section}{Appendix}

\section{Additional data details}\label{additional-data-details}

\section{Model details}\label{sec-model-details}

\subsection{Posterior predictive
check}\label{posterior-predictive-check}

In \textbf{?@fig-ppcheckandposteriorvsprior-1} we implement a posterior
predictive check. This shows\ldots{}

In \textbf{?@fig-ppcheckandposteriorvsprior-2} we compare the posterior
with the prior. This shows\ldots{}

\begin{Shaded}
\begin{Highlighting}[]
\CommentTok{\# \#| eval: true}
\CommentTok{\# \#| echo: false}
\CommentTok{\# \#| message: false}
\CommentTok{\# \#| warning: false}
\CommentTok{\# \#| label: fig{-}ppcheckandposteriorvsprior}
\CommentTok{\# \#| layout{-}ncol: 2}
\CommentTok{\# \#| fig{-}cap: "Examining how the model fits, and is affected by, the data"}
\CommentTok{\# \#| fig{-}subcap: ["Posterior prediction check", "Comparing the posterior with the prior"]}
\CommentTok{\# }
\CommentTok{\# pp\_check(first\_model) +}
\CommentTok{\#   theme\_classic() +}
\CommentTok{\#   theme(legend.position = "bottom")}
\CommentTok{\# }
\CommentTok{\# posterior\_vs\_prior(first\_model) +}
\CommentTok{\#   theme\_minimal() +}
\CommentTok{\#   scale\_color\_brewer(palette = "Set1") +}
\CommentTok{\#   theme(legend.position = "bottom") +}
\CommentTok{\#   coord\_flip()}
\end{Highlighting}
\end{Shaded}

\subsection{Diagnostics}\label{diagnostics}

\textbf{?@fig-stanareyouokay-1} is a trace plot. It shows\ldots{} This
suggests\ldots{}

\textbf{?@fig-stanareyouokay-2} is a hat plot. It shows\ldots{} This
suggests\ldots{}

\newpage

\section*{References}\label{references}
\addcontentsline{toc}{section}{References}

\phantomsection\label{refs}
\begin{CSLReferences}{1}{0}
\bibitem[\citeproctext]{ref-alexander2023}
Alexander, Rohan. 2023. \emph{Telling Stories with Data}.
\url{https://tellingstorieswithdata.com/}.

\bibitem[\citeproctext]{ref-burke1997}
Burke, Andrew E. 1997. {``Small Firm Start-up by Composers in the
Recording Industry.''} \emph{Small Business Economics} 9 (6): 463--71.
\url{https://doi.org/10.1023/A:1007968604929}.

\bibitem[\citeproctext]{ref-dodds2010}
Dodds, Peter Sheridan, and Christopher M. Danforth. 2010. {``Measuring
the {Happiness} of {Large}-{Scale} {Written} {Expression}: {Songs},
{Blogs}, and {Presidents}.''} \emph{Journal of Happiness Studies} 11
(4): 441--56. \url{https://doi.org/10.1007/s10902-009-9150-9}.

\bibitem[\citeproctext]{ref-citeLahman}
Friendly, Michael, Chris Dalzell, Martin Monkman, and Dennis Murphy.
2020. \emph{Lahman: Sean {``Lahman''} Baseball Database}.
\url{https://CRAN.R-project.org/package=Lahman}.

\bibitem[\citeproctext]{ref-rstanarm}
Goodrich, Ben, Jonah Gabry, Imad Ali, and Sam Brilleman. 2022.
{``Rstanarm: {Bayesian} Applied Regression Modeling via {Stan}.''}
\url{https://mc-stan.org/rstanarm/}.

\bibitem[\citeproctext]{ref-kim2021}
Kim, Seon Tae, and Joo Hee Oh. 2021. {``Music Intelligence: {Granular}
Data and Prediction of Top Ten Hit Songs.''} \emph{Decision Support
Systems} 145 (June): 113535.
\url{https://doi.org/10.1016/j.dss.2021.113535}.

\bibitem[\citeproctext]{ref-citeR}
R Core Team. 2023. \emph{R: A Language and Environment for Statistical
Computing}. Vienna, Austria: R Foundation for Statistical Computing.
\url{https://www.R-project.org/}.

\bibitem[\citeproctext]{ref-rosen1981}
Rosen, Sherwin. 1981. {``The {Economics} of {Superstars}.''} \emph{The
American Economic Review} 71 (5): 845--58.
\url{https://www.jstor.org/stable/1803469}.

\bibitem[\citeproctext]{ref-tidyverse}
Wickham, Hadley, Mara Averick, Jennifer Bryan, Winston Chang, Lucy
D'Agostino McGowan, Romain François, Garrett Grolemund, et al. 2019.
{``Welcome to the {tidyverse}.''} \emph{Journal of Open Source Software}
4 (43): 1686. \url{https://doi.org/10.21105/joss.01686}.

\end{CSLReferences}



\end{document}
